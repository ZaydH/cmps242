\documentclass[10pt]{beamer}
%\documentclass[handout,10pt]{beamer}  % Remove pauses and disable commands like "onslide"
\usetheme[
%%% options passed to the outer theme
%    hidetitle,           % hide the (short) title in the sidebar
%    hideauthor,          % hide the (short) author in the sidebar
%    hideinstitute,       % hide the (short) institute in the bottom of the sidebar
%    shownavsym,          % show the navigation symbols
%    width=2cm,           % width of the sidebar (default is 2 cm)
%    hideothersubsections,% hide all subsections but the subsections in the current section
%    hideallsubsections,  % hide all subsections
left               % right of left position of sidebar (default is right)
%%% options passed to the color theme
%    lightheaderbg,       % use a light header background
]{UCSCsidebar}

\usepackage{environ}
\usepackage{tikz}
\usepackage{listings} % Used for printing source code in papers
\usepackage{pgfplots}\pgfplotsset{compat=newest} % Used to create graphs
\usepackage[utf8]{inputenc}
\usepackage[english]{babel}
\usepackage[T1]{fontenc}
% Or whatever. Note that the encoding and the font should match. If T1
% does not look nice, try deleting the line with the fontenc.
\usepackage{helvet}
\usepackage{makecell} % Used to create thick links in tables
\usepackage{multirow} % Allows merging rows or columns in a table

%%%%%%%%%%%%%%%%%%%%%%%%%%%%%%%%%%%%%%%%%%
%%   Scales tikz images to slide size   %%
%%%%%%%%%%%%%%%%%%%%%%%%%%%%%%%%%%%%%%%%%%
\makeatletter
\newsavebox{\measure@tikzpicture}
\NewEnviron{scaletikzpicturetowidth}[1]{%
  \def\tikz@width{#1}%
  \def\tikzscale{1}\begin{lrbox}{\measure@tikzpicture}%
    \BODY
  \end{lrbox}%
  \pgfmathparse{#1/\wd\measure@tikzpicture}%
  \edef\tikzscale{\pgfmathresult}%
  \BODY
}
\makeatother
%%%%%%%%%%%%%%%%%%%%%%%%%%%%%%%%%%%%%%%%%%



\title[Trump-ian Speech Generation]% optional, use only with long paper titles
{\textbf{Make RNNs Great Again}}

\subtitle{Character-Level Sequence Generation in the Style of Donald Trump}  % could also be a conference name

\date{December 5, 2017}

\author[Sherman \& Hammoudeh] % optional, use only with lots of authors
{
  \href{mailto:bcsherma@ucsc.edu}{{Benjamin Sherman}}\\
  \&\\  
  \href{mailto:zayd@ucsc.edu}{{Zayd Hammoudeh}}
}
% - Give the names in the same order as they appear in the paper.
% - Use the \inst{?} command only if the authors have different
%   affiliation. See the beamer manual for an example

\institute[
%  {\includegraphics[scale=0.2]{SJSU_segl}}\\ %insert a company, department or university logo
Dept.\ of Computer Science\\
UC, Santa Cruz\\
] % optional - is placed in the bottom of the sidebar on every slide
{% is placed on the title page
  Dept.\ of Computer Science\\
  University of California, Santa Cruz\\
  
  %there must be an empty line above this line - otherwise some unwanted space is added between the university and the country (I do not know why;( )
}


\begin{document}
  % the titlepage
  {\begin{frame}[plain,noframenumbering]{}{} % the plain option removes the sidebar and header from the title page
    \titlepage
  \end{frame}}
  %%%%%%%%%%%%%%%%

	% introduction slide
	\begin{frame}{Introduction}
		
		\begin{itemize}
			\item Character level RNNs (recurrent neural networks) learn generative models for text at the character level
			
			\item Our aim was to construct an RNN that would train on speeches made by Donald Trump and then be able to generate Trump-esque speech
			
		\end{itemize}
		
	\end{frame}
	
	% overview of what character level RNNs are
	\begin{frame}{Character Level RNNs}
		
		\begin{itemize}
			\item Given a sequence of characters, a character level RNN learns a probability distribution over the possible subsequent characters
			
			\item Example: if you give as input ``We will build a great wal'', the RNN should return a distribution $p$ over a vocabulary of characters $V$ s.t. $p(\text{`l'})$ is large
			
			\item Given some text, you can repeatedly choose a subsequent character based on the distribution produced by the network given the previous $L$ characters
			
		\end{itemize}
		
	\end{frame}
	
	% our network architecture for training
	\begin{frame}{High Level Architecture}
		
		% I think this slide should just the pic of our architecture from the 
		
	\end{frame}
	
	% decision engine/generation process
	\begin{frame}{Decision Engine}
		
		\begin{itemize}
			\item Greedy Sampling
			\item Dropout
			\item Random
			\item Blowup % meaning exponentiate and re-normalize the distribution
		\end{itemize}
		
	\end{frame}
	
\end{document}
