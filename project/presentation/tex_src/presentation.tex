\documentclass[10pt,aspectratio=169]{beamer}
%\documentclass[handout,10pt]{beamer}  % Remove pauses and disable commands like "onslide"
\usetheme[
%%% options passed to the outer theme
    hidetitle,           % hide the (short) title in the sidebar
%    hideauthor,          % hide the (short) author in the sidebar
%    hideinstitute,       % hide the (short) institute in the bottom of the sidebar
%    shownavsym,          % show the navigation symbols
%    width=2cm,           % width of the sidebar (default is 2 cm)
%    hideothersubsections,% hide all subsections but the subsections in the current section
%    hideallsubsections,  % hide all subsections
left               % right of left position of sidebar (default is right)
%%% options passed to the color theme
%    lightheaderbg,       % use a light header background
]{UCSCsidebar}

\usepackage{environ}
\usepackage{tikz}
\usetikzlibrary{matrix, positioning, calc, shadows, decorations.markings, arrows.meta} % decorations.markings is used for the bus symbol on the arrow.
% arrows.meta allows changing the arrow heads.
\usepackage{listings} % Used for printing source code in papers
\usepackage{pgfplots}\pgfplotsset{compat=newest} % Used to create graphs
\usepackage[utf8]{inputenc}
\usepackage[english]{babel}
\usepackage[T1]{fontenc}
% Or whatever. Note that the encoding and the font should match. If T1
% does not look nice, try deleting the line with the fontenc.
\usepackage{helvet}
\usepackage{makecell} % Used to create thick links in tables
\usepackage{multirow} % Allows merging rows or columns in a table

%%%%%%%%%%%%%%%%%%%%%%%%%%%%%%%%%%%%%%%%%%
%%   Scales tikz images to slide size   %%
%%%%%%%%%%%%%%%%%%%%%%%%%%%%%%%%%%%%%%%%%%
\makeatletter
\newsavebox{\measure@tikzpicture}
\NewEnviron{scaletikzpicturetowidth}[1]{%
  \def\tikz@width{#1}%
  \def\tikzscale{1}\begin{lrbox}{\measure@tikzpicture}%
    \BODY
  \end{lrbox}%
  \pgfmathparse{#1/\wd\measure@tikzpicture}%
  \edef\tikzscale{\pgfmathresult}%
  \BODY
}
\makeatother
%%%%%%%%%%%%%%%%%%%%%%%%%%%%%%%%%%%%%%%%%%



\title[Trump-ian Speech Generation]% optional, use only with long paper titles
{\textbf{Make RNNs Great Again}}

\subtitle{Character-Level Sequence Generation in the Style of Donald Trump}  % could also be a conference name

\date{December 5, 2017}

\author[Sherman \& Hammoudeh] % optional, use only with lots of authors
{
  \href{mailto:bcsherma@ucsc.edu}{Benjamin Sherman}\\
  \&\\  
  \href{mailto:zayd@ucsc.edu}{Zayd Hammoudeh}
}
% - Give the names in the same order as they appear in the paper.
% - Use the \inst{?} command only if the authors have different
%   affiliation. See the beamer manual for an example

\institute[
%  {\includegraphics[scale=0.2]{SJSU_segl}}\\ %insert a company, department or university logo
Dept.\ of Computer Science\\
UC, Santa Cruz\\
] % optional - is placed in the bottom of the sidebar on every slide
{% is placed on the title page
  Dept.\ of Computer Science\\
  University of California, Santa Cruz\\
  
  %there must be an empty line above this line - otherwise some unwanted space is added between the university and the country (I do not know why;( )
}





\begin{document}
  % the titlepage
  {\begin{frame}[plain,noframenumbering]{}{} % the plain option removes the sidebar and header from the title page
    \titlepage
  \end{frame}}
  %%%%%%%%%%%%%%%%


	% introduction slide
  \section{Introduction}
  \begin{frame}{\textbf{Quiz}}
  \onslide<2->{
    \textbf{Two Quotes}: One computer-generated and one from Donald Trump.  Can you identify the real one?
  }
  \vfill
  \begin{itemize}
    \onslide<3-3>{
      \item \textbf{Quote \#1}: \textit{You look at the nuclear deal,} thing that real really bothers me, it would have been so easy and its not - as important as these lives are - nuclear is so powerful. My uncle explained that to me many, many years ago...
    }
    \vfill
    \only<4->{
      \item \textbf{Quote \#2}: \textit{You look at the nuclear deal,} and it’s going to be great for most of my plan in a country that they don’t know it. We have to be so good and it’s a movement and we will make America strong again. Thank you. Thank you...
    }
  \end{itemize}
  \end{frame}

	% introduction slide
	\begin{frame}{\textbf{Project Objectives}}
		
		\begin{itemize}
      \setlength\itemsep{1.5em}
      \onslide<+->{
			 \item \textbf{Primary Objective}: Develop a character-level neural network that can generate text in the style of Donald Trump.
      }
  
      \onslide<+->{
        \item \textbf{Secondary Objectives}:
      	\begin{enumerate}
          \setlength\itemsep{1.5em}
          \item Develop and compare novel \textit{decision engine} algorithms for character selection.
          \item Improve short-sequence generation through multi-length training.
        \end{enumerate}
      }
    \end{itemize}
	\end{frame}
	
	% overview of what character level RNNs are
	\begin{frame}{\textbf{Quick Review of Character-Level RNNs}}
		
		\begin{itemize}
			\onslide<+->{
				\item Given a sequence of characters, a character-level RNN learns a probability distribution over the possible subsequent characters
			}
      \vfill
			\onslide<+->{
				\item \textbf{Example:} if you give as input ``We will build a great wal'', the RNN should return a distribution $p$ over a vocabulary of characters $V$ s.t. $p(\text{`l'})$ is large
			}
			\vfill
			\onslide<+->{
				\item Given some text, you can repeatedly choose a subsequent character based on the distribution produced by the network given the previous $L$ characters
			}
			
		\end{itemize}
		
	\end{frame}
	
	\section{Learner Architecture}
  \begin{frame}{\textbf{Base} Learner Architecture}
    \centering
    \tikzset{
      sigmoid/.style={path picture= {
          \begin{scope}[x=1pt,y=10pt]
            \draw plot[domain=-6:6] (\x,{1/(1 + exp(-\x))-0.5});
          \end{scope}
        }
      },
      every neuron/.style={
        circle,
        draw,
        fill=white,
        minimum size=0.6cm,
      },
      neuron output/.style={
        every neuron,
        sigmoid,
      },
      matrix multiply/.style={
        every neuron,
        font={\Large $\times$}
      },
      every netbox/.style={
        rectangle,
        draw,
        text centered,
        fill=white,
        text width = 1.3cm,
        minimum width = 1.5cm,
        minimum height = 1.5cm,
        rounded corners,
      },
      every left delimiter/.style={xshift=1ex},
      every right delimiter/.style={xshift=-1ex},
      bmatrix/.style={matrix of math nodes,
        inner sep=0pt,
        left delimiter={[},
        right delimiter={]},
        nodes={anchor=center, inner sep=.3333em},
      },
      decoration={
        markings,
        mark= at position 0.5 with {\node {/};}
      }
    }
    
    \begin{tikzpicture}[x=1.5cm, y=1.5cm, >=stealth] 
    
    % Must define nodes in the list before they can be used.
    \node [neuron output,  drop shadow] (softmax) {};
    \node [every netbox, left of=softmax, xshift=-0.35in, drop shadow] (feedforward) {FF};
    \node [every netbox, left of=feedforward, xshift=-0.45in, drop shadow] (lstmbox) {LSTM};
    \node [every neuron, left of=lstmbox, xshift=-0.25in, drop shadow] (matmul) {\Large$\times$};
    \matrix (oneHotFirst) [bmatrix, below left of = matmul, xshift=-0.2in, yshift = -0.4in] {\tiny
      0 \\
      1 \\
      0 \\[-1.5ex]
      \vdots \\
      0\\};
    \matrix (oneHotSecond) [bmatrix, left of = oneHotFirst, xshift=0.1in] {\tiny
      0 \\
      1 \\
      0 \\[-1.5ex]
      \vdots \\
      0\\};
    \node [left of=oneHotSecond, xshift=0.1in] (oneHotContinue) {\Large $\cdots$};
    \matrix (oneHotLast) [bmatrix,left of = oneHotContinue, xshift=0.05in] {\tiny
      1 \\
      0 \\
      0 \\[-1.5ex]
      \vdots \\
      0\\};
    \node [above of=oneHotContinue, text centered, xshift=0.1in, yshift=0.16in] (onehotlabel) {One-Hot Vector Inputs};
    
    \matrix (embedding) [bmatrix, above left of = matmul, xshift=-0.7in, yshift=0.5in,ampersand replacement=\&] {
      w_{1,1} \& \cdots \& w_{1,|v|}\\[-1ex]
      \vdots \& \ddots \& \vdots\\
      w_{d,1} \& \cdots \& w_{d,|v|}\\};
    \node [above of=embedding, yshift=0.05in] (embeddinglabel) {Embedding Matrix};
    
    % Draw arrows
    \draw [->,thick,line width=0.4mm] (embedding) -- (matmul);
    \draw [->,thick,line width=0.4mm] (oneHotFirst) -- (matmul);
    \draw [->,thick,line width=0.4mm] (matmul) -- (lstmbox);
    \draw [->,thick,line width=0.4mm] (lstmbox) -- (feedforward);
    \draw [->,thick,line width=0.4mm] (lstmbox) to[out=30, in=150, looseness=3] (lstmbox); % Numbers represent location on the unit circle.  0 is due east, 90 due north, 180 due west, and 270 due south.
    \draw [->,thick,line width=0.4mm] (feedforward.north east) -- (softmax);
    \draw [->,thick,line width=0.4mm] (feedforward) -- (softmax);
    \draw [->,thick,line width=0.4mm] (feedforward.south east) -- (softmax);
    \draw [->,thick,line width=0.4mm,postaction={decorate}] (softmax) -- node[below=1pt] {$|v|$} ++(0.9,0);
    \end{tikzpicture}
    
  \end{frame}
  
  
  
  \begin{frame}{\textbf{Learner Architecture -- Summary}}
  
    \begin{itemize}
      \setlength\itemsep{0.8em}
      \onslide<+->{
        \item \textbf{Five Primary Stages}:
        \begin{itemize}
          \setlength\itemsep{0.4em}
          \item One-Hot \textit{Character} Encoding
          \item Embedding Matrix
          \item Multi-Layer LSTM
          \item Feed-Forward Network
          \item Softmax Layer
        \end{itemize}
      }
  
      \onslide<+->{
        \item \textbf{One-Hot \& Softmax Dimension}: \textasciitilde95 Characters
      }
    
      \onslide<+->{
        \item \textbf{LSTM}: 
        \begin{itemize}
          \setlength\itemsep{0.4em}
          \item Two Layers
          \item Hidden Layer Width: 128
          \item Dropout -- Surprisingly important! (\textit{More details to come})
        \end{itemize}
      }
    
      \onslide<+->{
        \item \textbf{Feed-Forward Network}:
        \begin{itemize}
          \item One Hidden Layer with 256 Neurons
        \end{itemize}
      }
    \end{itemize}  
  \end{frame}

  \section{Training Overview}
  \begin{frame}{\textbf{Discussion on Character-Level Text Generation}}
    \begin{itemize}
      \setlength\itemsep{1em}
      \onslide<+->{
        \item \textbf{Question: }\textit{Wouldn't a word-level RNN be better?}
      }
      \onslide<+->{
        \item \textbf{Short Answer:} Word-level is not practical.  
      }
      \onslide<+->{
        \item \textbf{Long Answer:}
        \begin{itemize}
          \setlength\itemsep{1em}
          \onslide<+->{
            \item Too many words (i.e., classes)
          }
          \onslide<+->{
            \item Limited hardware availability
          }
          \onslide<+->{
            \item Limited training time
          }
        \end{itemize}
      }
      \onslide<+->{
        \item \textbf{Question:} \textit{Is character-level text generation ideal?}
      }
      \onslide<+->{
        \item \textbf{No.} We do \textbf{not} expect a character-level RNN to create perfectly coherent text.  
        \begin{itemize}
          \item It will only successfully mimic short phrases or at most a single paragraph.
        \end{itemize}
      }
    \end{itemize}
  
  \end{frame}
  
  \section{Training Overview}
  \begin{frame}{\textbf{Overview of the Training Dataset \& Procedure}}

  \begin{itemize}
    \item Datasets
    \begin{itemize}
      \item Approximately 115~speeches
      \item Basic Statistics:
      \begin{itemize}
        \item >365,000 Words
        \item >2,000,000 Training Sequences
      \end{itemize}
    \end{itemize}
    \vfill
    \item Speeches Only
    \vfill
    \item New Innovation: Variable length sequence training
    \begin{itemize}
      \item Rather than training only the maximum sequence length, we train intermediary sequence lengths to ensure quality outputs even on short sequences.
    \end{itemize}
  \end{itemize}

  
  \end{frame}
  
  
  \section{Text Generation}
  \centering
  \begin{frame}{\textbf{Text Generation Architecture}}
  \tikzset{
    sigmoid/.style={path picture= {
        \begin{scope}[x=1pt,y=10pt]
          \draw plot[domain=-6:6] (\x,{1/(1 + exp(-\x))-0.5});
        \end{scope}
      }
    },
    every neuron/.style={
      circle,
      draw,
      fill=white,
      minimum size=0.6cm,
    },
    neuron output/.style={
      every neuron,
      sigmoid,
    },
    matrix multiply/.style={
      every neuron,
      font={\Large $\times$}
    },
    every netbox/.style={
      rectangle,
      draw,
      text centered,
      fill=white,
      text width = 1.3cm,
      minimum width = 1.5cm,
      minimum height = 1.5cm,
      rounded corners,
    },
    every left delimiter/.style={xshift=1ex},
    every right delimiter/.style={xshift=-1ex},
    bmatrix/.style={matrix of math nodes,
      inner sep=0pt,
      left delimiter={[},
      right delimiter={]},
      nodes={anchor=center, inner sep=.3333em},
    },
    decoration={
      markings,
      mark= at position 0.5 with {\node {/};}
    }
  }
  
  \begin{tikzpicture}[x=1.5cm, y=1.5cm, >=stealth] 
  
  % Must define nodes in the list before they can be used.
  \onslide<3->{
    \node [every netbox, drop shadow] (output) {\color{ucscBlue} Decision Engine};
  }
  \node [neuron output, left of=output, xshift=-0.35in,  drop shadow] (softmax) {};
  \node [every netbox, left of=softmax, xshift=-0.35in, drop shadow] (feedforward) {FF};
  \node [every netbox, left of=feedforward, xshift=-0.45in, drop shadow] (lstmbox) {LSTM};
  \node [every neuron, left of=lstmbox, xshift=-0.25in, drop shadow] (matmul) {\Large$\times$};
  \matrix (oneHotFirst) [bmatrix, below left of = matmul, xshift=-0.2in, yshift = -0.4in] {\tiny
    0 \\
    1 \\
    0 \\[-1.5ex]
    \vdots \\
    0\\};
  \matrix (oneHotSecond) [bmatrix, left of = oneHotFirst, xshift=0.1in] {\tiny
    0 \\
    1 \\
    0 \\[-1.5ex]
    \vdots \\
    0\\};
  \node [left of=oneHotSecond, xshift=0.1in] (oneHotContinue) {\Large $\cdots$};
  \matrix (oneHotLast) [bmatrix,left of = oneHotContinue, xshift=0.05in] {\tiny
    1 \\
    0 \\
    0 \\[-1.5ex]
    \vdots \\
    0\\};
  \node [above of=oneHotContinue, text centered, xshift=0.1in, yshift=0.16in] (onehotlabel) {One-Hot Vector Inputs};
  
  \matrix (embedding) [bmatrix, above left of = matmul, xshift=-0.7in, yshift=0.5in,ampersand replacement=\&] {
    w_{1,1} \& \cdots \& w_{1,|v|}\\[-1ex]
    \vdots \& \ddots \& \vdots\\
    w_{d,1} \& \cdots \& w_{d,|v|}\\};
  \node [above of=embedding, yshift=0.05in] (embeddinglabel) {Embedding Matrix};
  
  % Draw arrows
  \draw [->,thick,line width=0.4mm] (embedding) -- (matmul);
  \draw [->,thick,line width=0.4mm] (oneHotFirst) -- (matmul);
  \draw [->,thick,line width=0.4mm] (matmul) -- (lstmbox);
  \draw [->,thick,line width=0.4mm] (lstmbox) -- (feedforward);
  \draw [->,thick,line width=0.4mm] (lstmbox) to[out=30, in=150, looseness=3] (lstmbox); % Numbers represent location on the unit circle.  0 is due east, 90 due north, 180 due west, and 270 due south.
  \draw [->,thick,line width=0.4mm] (feedforward.north east) -- (softmax);
  \draw [->,thick,line width=0.4mm] (feedforward) -- (softmax);
  \draw [->,thick,line width=0.4mm] (feedforward.south east) -- (softmax);
  \draw [->,thick,line width=0.4mm,postaction={decorate}] (softmax) -- node[below=1pt] {$|v|$} (output);
  \onslide<2-3>{
    \draw [{Latex[scale=0.8]}-,line width=0.4mm,darkGreen] (oneHotFirst.south) -- ++(0,-0.25);
    \draw [{Latex[scale=0.8]}-,line width=0.4mm,darkGreen] (oneHotSecond.south) -- ++(0,-0.25);
    \draw [{Latex[scale=0.8]}-,line width=0.4mm,darkGreen] (oneHotLast.south) -- ++(0,-0.25);
    \node [below of=oneHotContinue, text centered, xshift=0.1in, yshift=-0.35in] (seedTextInput) {\color{darkGreen} \textbf{Seed Text}};
  }
  \onslide<3->{
    \draw [->,line width=0.4mm,ucscBlue] (output) -- ++(0.9,0);
  }
  \onslide<5->{
    \draw [-{Latex[scale=0.8]},line width=0.4mm,dashed,red] (output.east) -| ++(0.15,0) -| ++(0,-2.3) -| (oneHotLast.south);
    \draw [{Latex[scale=0.8]}-,line width=0.4mm,dashed,red] (oneHotFirst.south) to[bend left] (oneHotSecond.south);
    \draw [{Latex[scale=0.8]}-,line width=0.4mm,dashed,red] (oneHotSecond.south) to[bend left] ++(-0.4,0);
    \draw [-{Latex[scale=0.8]},line width=0.4mm,dashed,red] (oneHotLast.south) to [bend right] ++(0.4,0);
  }
  \end{tikzpicture}
  
  \end{frame}

	% decision engine/generation process
  \section{Decision Engine}
	\begin{frame}{\textbf{Decision Engine}}
		\begin{itemize}
			\onslide<+->{
				\item \textbf{Greedy }\textit{Always take the most probable character}
				
				\begin{itemize}
					\onslide<+->{\item Always makes a confident choice}
					\onslide<+->{\item Leads to looping behavior }
					\onslide<+->{\item \textit{``\textcolor{ucscBlue}{The media is so dishonest.} They want to stop the people of the world. I want to stop the people of the world. I want to stop the people of the world. I want...''}}
				\end{itemize}
			}
			
			
			\onslide<+->{
				\item \textbf{Random }\textit{Select a random character according to the given distribution}
				
				\begin{itemize}
					\onslide<+->{\item Avoids getting into loops}
					\onslide<+->{\item Has a non-zero probability of doing something ridiculous}
					\onslide<+->{\item \textit{``\textcolor{ucscBlue}{The media is so dishonest.} And thank you. I trace it. We can change it. We don’t worts out.''}}
				\end{itemize}
			}
			
			\onslide<+->{
				\item \textbf{Top-k }\textit{Make a random choice amongst the $k$ most probable characters according to the sub-distribution}
				
				\begin{itemize}
					\onslide<+->{\item Throws out ridiculous random choices}
					\onslide<+->{\item \textit{``\textcolor{ucscBlue}{The media is so dishonest.} And they don’t know where you see it. I'm going to bring back the world.''}}
				\end{itemize}
			}
			
		\end{itemize}
	\end{frame}
	
	\begin{frame}{\textbf{Decision Engine} Contd.}
		
		\begin{itemize}
			\onslide<+->{\item \textbf{Random-Start + Greedy Finish} \textit{Make random choices for the first character of a word, then greedily finish each word}}
			
				\begin{itemize}
					\onslide<+->{\item Gets us out of infinite loops}
					\onslide<+->{\item Does not mangle characters within words}
				\end{itemize}
			\vfill
			\onslide<+->{\item \textbf{Boosting Lopsidedness} \textit{Exponentiate the distribution then re-normalize}}
				
				\begin{itemize}
					\onslide<+->{\item Boosts the chance of a making a more probable choice}
				\end{itemize}
			\vfill
		 	\onslide<+->{\item \textbf{Sample with Dropout}}
					
				\begin{itemize}
					\onslide<+->{\item Randomly change the network}
				\end{itemize}
			\vfill
			\onslide<+->{\item \textbf{We can combine these with the algorithms above to create a wide variety of sampling methods}}
		\end{itemize}
	\end{frame}

	% decision engine/generation process
  \section{Conclusions}
  \begin{frame}{\textbf{Summary and Future Work}}
    \begin{itemize}
      \item \textbf{Summary}: We developed a character-level RNN that generated text in the style of President Donald Trump.
      \vfill
      \item \textbf{Future Work:}
      \begin{itemize}
        \onslide<+->{
          \item \textbf{New Idea}: Character-Level Generation, Word-Level Decisions
        }
        \vfill
        \onslide<+->{
          \item \textbf{\textbf{Training Improvements}}: Given more time, we believe we could train better models.
          \begin{itemize}
            \item Train longer sequences
            \item Train entire speeches as a single unit
            \item Train additional epochs (each additional epoch is about 1~hour of CPU time)
          \end{itemize}
        }
        \vfill
        \onslide<+->{\item \textbf{Stretch Goal}: A chat bot so you can feel what it is like to have a conversation with Donald Trump.
        }
      \end{itemize}
    \end{itemize}
  \end{frame}

  \begin{frame}{\textbf{Download the Source Code}}
  
    Our full source code is available at:
    
    \begin{center}
      \href{https://github.com/ZaydH/trump_char_rnn}{\textbf{\color{ucscBlue} https://github.com/ZaydH/trump\_char\_rnn}}
    \end{center}
  
  \end{frame}


  %% Optional Table of Contents
  \begin{frame}{\textbf{Summary of Topics}}{}
    \tableofcontents
  \end{frame}
  %%%%%%%%%%%%%%%%
	
\end{document}
