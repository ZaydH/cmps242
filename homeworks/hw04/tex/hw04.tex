\documentclass{report}

\usepackage{fullpage}
\usepackage[skip=4pt]{caption} % ``skip'' sets the spacing between the figure and the caption.
\usepackage{pgfplots}   % Needed for plotting
\usepackage{amsmath}    % Allows for piecewise functions using the ``cases'' construct
%\usepackage{mathrsfs}   % Use the ``\mathscr'' command in an equation.

\usepackage[obeyspaces,spaces]{url} % Used for typesetting with the ``path'' command
\usepackage[hidelinks]{hyperref}   % Make the cross references clickable hyperlinks
\usepackage[bottom]{footmisc} % Prevents the table going below the footnote
\usepackage{nccmath}    % Needed in the workaround for the ``aligncustom'' environment
\usepackage{amssymb}    % Used for black QED symbol   
\usepackage{bm}    % Allows for bolding math symbols.
\usepackage{tabto}     % Allows to tab to certain point on a line

\newcommand{\hangindentdistance}{1cm}
\setlength{\parindent}{0pt}
\setlength{\leftskip}{\hangindentdistance}
\setlength{\hangafter}{1}


% Set up page formatting
\usepackage{todonotes}
\usepackage{fancyhdr} % Used for every page footer and title.
\pagestyle{fancy}
\fancyhf{} % Clears both the header and footer
\renewcommand{\headrulewidth}{0pt} % Eliminates line at the top of the page.
\fancyfoot[LO]{CMPS242 \textendash{} Homework \#4} % Left
\fancyfoot[CO]{\thepage} % Center
\fancyfoot[RO]{Zayd Hammoudeh} %Right

% Change interline spacing.
\renewcommand{\baselinestretch}{1.1}
\newenvironment{aligncustom}
{ \csname align*\endcsname % Need to do this instead of \begin{align*} because of LaTeX bug.
    \centering
}
{
  \csname endalign*\endcsname
}
%--------------------------------------------------


\title{\textbf{CMPS242 Homework \#4}}
\author{Zayd Hammoudeh}

%---------------------------------------------------%
% Define the Environments for the Problem Inclusion %
%---------------------------------------------------%
\newcounter{problemCount} 
\setcounter{problemCount}{0} % Reset the subproblem counter
\newenvironment{problemshell}{
  \par%
  \medskip
  \leftskip=0pt\rightskip=0pt%
}
{
  \par\medskip
}
\newenvironment{problem}
{%
  \stepcounter{problemCount}
  \begin{problemshell}
    \noindent \textit{Problem \#\arabic{problemCount}} \\
    \bfseries  
}
{
  \end{problemshell}
}


\newcommand{\problemspace}{\\[0.4em]}
\newcommand{\sign}{\text{\normalfont sign}}
\newcommand{\w}{\mathbf{w}}
\newcommand{\xvec}{\mathbf{x}}
\newcommand{\wx}{\w\cdot\xvec}
\newcommand{\norm}[1]{\lVert\mathbf{#1}\rVert^{2}}
\newcommand{\yhat}{\hat{y}}
\newcommand{\T}{\text{T}}


\begin{document}
  \maketitle
  
  \begin{problem}
    Consider $1$-dimensional linear regression.
    \problemspace
    First compute the optimum solution,~$w^{*}$, for a batch of examples,~${(x_i,y_i)}$, ${1 \leq i \leq n}$,~ie the weight that minimizes the total loss on examples:~${L(w)=\sum_{i=1}^n(wx_{i}-y_{i})^2}$.
    \problemspace
    Assume labels are expensive (see lecture~7).  You are given only one of the labels~$y_{i}$.  Compute the optimal solution~$w_i$ based on a single example~$(x_i,y_i)$.
    \problemspace
    Show that if~$i$ is chosen wrt the distribution~$\frac{x_{i}^{2}}{\sum_{j}x_{j}^{2}}$, then the expected loss of~$w_{i}$ on all examples is twice the optimum ie
    \[\mathbf{E}[L(w_{i}^{*})] = 2L(w^{*}), \]
    \noindent
    when all~$x_i$ are non-zero.
    \problemspace
    Hint: First check the above equation on Octave or Matlab on some random data.  Make your solution as simple as you can.
  \end{problem}
  
  For a set of $n$~ordered pairs~$(x_i,y_i)$, the squared loss function~$L$ is given by
  
  \begin{equation}
    L(w)=\sum_{j} (w x_{j}-y_{j})^2\textrm{.}
  \end{equation}

  The optimal weight vector,~$w^{*}$ can be found by taking the derivative and setting it equal to 0.  Hence,
  
  \begin{align}
    \frac{\partial L(w)}{\partial w}=0&=\frac{\partial}{\partial w}\sum_{j} (w^{*} x_{j}-y_{j})^2\\
    0&=\sum_{j} 2(w^{*} x_{j}-y_{j})x_{j}\\
    \sum_{j} x_{j}y_{j} &= w^{*} \sum_{j} x_{j}^2\\
    w^{*}&=\frac{\sum_{j}x_{j}y_{j}}{\norm{x}}=\frac{\mathbf{x}\cdot\mathbf{y}}{\norm{x}} \textrm{.}
  \end{align}
  
  For a single example, the weight would just be:
  
  \begin{equation}
    w_{i}^{*}=\frac{y_{i}}{x_{i}}
  \end{equation}
  
  If our selection of $x_i$ was deterministic, then it would perform poorly against an adversary. Hence, it needs to be probabilistic.  The probability distribution specified in the problem was
  
  \begin{align}
    P(X=x_{i})&=\frac{x_{i}^{2}}{\sum_{j} x_{j}^{2}}\\
    &=\frac{x_{i}^{2}}{\norm{x}}\textrm{.}
  \end{align}

  Therefore, the expected loss would be:
  
  \begin{align}
    \mathbf{E}\left[ L(w_{i}^{*}) \right] &= \sum_{i} \Pr(X=x_i)L(w_{i}^{*})\\
    &= \sum_{i} \frac{x_{i}^2}{\norm{x}}L(w_{i}^{*})\\
    &= \sum_{i} \frac{x_{i}^2}{\norm{x}}\sum_{j} (w_{i}^{*} x_{j}-y_{j})^{2}\\
    &= \sum_{i} \frac{x_{i}^2}{\norm{x}}\sum_{j} \left( \left(w_{i}^{*}\right)^{2}x^{2}_{j}-2w_{i}^{*}x_{j}y_{j}+y^{2}_{j}\right)\\
    &= \sum_{i} \left(  \frac{x_{i}^2}{\norm{x}}\sum_{j} \left( \left(w_{i}^{*}\right)^{2}x^{2}_{j}\right)-\frac{2}{\norm{x}}\sum_{j}\left(w_{i}^{*}x_{i}^{2}x_{j}y_{j}\right) +\frac{x_{i}^2}{\norm{x}}\sum_{j} y^{2}_{j} \right)\\
    &= \sum_{i} \left(  \frac{x_{i}^2}{\norm{x}}\sum_{j} \left( \frac{y^{2}_{i}}{x^{2}_{i}}x^{2}_{j}\right) -  \frac{2}{\norm{x}}\sum_{j}\left(\frac{y_{i}}{x_{i}}x_{i}^{2}x_{j}y_{j}\right)   +\frac{x_{i}^2}{\norm{x}}\sum_{j} y^{2}_{j} \right)\\
    &= \sum_{i} \left(  y^{2}_{i}- \frac{2}{\norm{x}}\sum_{j}\left(y_{i}x_{i}x_{j}y_{j}\right) +\frac{x_{i}^2}{\norm{x}}\norm{y} \right)\\
    &= \norm{y} +\frac{\norm{x}}{\norm{x}}\norm{y} -\sum_{i}\frac{2}{\norm{x}}\sum_{j}\left(y_{i}x_{i}x_{j}y_{j}\right) \\
    &= 2\norm{y} - \frac{2}{\norm{x}}\sum_{i}\sum_{j}\left(y_{i}y_{j}x_{i}x_{j}\right)\label{eq:expectedLoss}
  \end{align}
  
  We can then find the similar relation for~$w^{*}$.
  
  \begin{align}
    L(w^{*}) &= \sum_{i} (w^{*} x_{i}-y_{i})^2\\
    &= \sum_{i} \left( \left(w^{*}\right)^{2}x_{i}^{2}-2w^{*}x_{i}y_{i}+y^{2}_{i}\right)\\
    &= \norm{y} + \sum_{i} \left( \frac{\left(\sum_{j}x_{j}y_{j}\right)^{2}}{\left(\norm{x}\right)^2}x_{i}^{2}-2\frac{\left(\sum_{j}x_{j}y_{j}\right)}{\norm{x}}x_{i}y_{i}\right)\\
    &= \norm{y} +\frac{\left(\sum_{j}x_{j}y_{j}\right)^{2}}{\left(\norm{x}\right)^2}\norm{x} -\frac{2}{\norm{x}}\sum_{i}\sum_{j}y_{i}y_{j}x_{i}x_{j}\\
    &= \norm{y} +\frac{(\sum_{j}x_{j}y_{j})(\sum_{i}x_{i}y_{i})}{\norm{x}} -\frac{2}{\norm{x}}\sum_{i}\sum_{j}y_{i}y_{j}x_{i}x_{j}\\
    &= \norm{y} + \frac{1}{\norm{x}}\sum_{i}\sum_{j}y_{i}y_{j}x_{i}x_{j}-\frac{2}{\norm{x}}\sum_{i}\sum_{j}y_{i}y_{j}x_{i}x_{j}\\
    &= \norm{y} - \frac{1}{\norm{x}}\sum_{i}\sum_{j}\left(y_{i}y_{j}x_{i}x_{j}\right)\label{eq:optimalLoss}
  \end{align}
  
  Hence, from Eq.~\eqref{eq:expectedLoss} and Eq.~\eqref{eq:optimalLoss}, it is clear that 
  
  \[\mathbf{E}[L(w_{i}^{*})] = 2L(w^{*}), \]
  
  \noindent
  completing the proof.

  %---------------------------------------------------%
  \newpage
  \begin{problem}
    Compute all the derivatives using Backpropagation for a 3-layer neural net with one output when the transfer function is the cumulative Gaussian density
    \[\Phi(a) = \int_{-\infty}^{a}\frac{1}{\sqrt{2\pi}} \exp\left( -\frac{z^2}{2}\right) dz\]
    \noindent
    and the output node is the square loss.  Assume the node of the hidden layer as well as the output node each have a bias term.  Compute the derivatives of the loss wrt the weights between the 2nd~and 3rd~layer and the 1st~and 2nd~layer as well as the derivatives of the loss wrt the bias terms.
    \problemspace
    Hint: First produce a writeup when the transfer function is the sigmoid and then modify it.
  \end{problem}

  \textbf{Notation}: 
  \begin{itemize}
    \item $\yhat$ -- Predicted output value.
    \item $x_{j,k}^{l}$ -- Input to node~$j$ in layer~$l$ from node~$k$ in layer~$l-1$,
    \item $\xvec_{j}^{l}$ -- Vector notation for all neuron inputs from layer~$l-1$ to node~$j$ in layer~$l$.
    \item $w_{j,k}^{l}$ -- Weight for connection between the ${k^{\text{th}}}$~neuron in ${(l-1)^{\text{th}}}$~layer to the ${j^{\text{th}}}$~neuron in the ${l^{\text{th}}}$~layer.
    \item $\w_{j}^{l}$ -- Vector notation for all connection weights between layer~$l-1$ and node~$j$ in layer~$l$.
    \item $b_{j}^{l}$ -- Bias term for the ${j^{\text{th}}}$~neuron in the $l^{\text{th}}$~layer.
    \item $\Phi(a)$ -- Transfer function for the neurons.  It is given as:
    \begin{equation}
      \Phi(a) :=\int_{-\infty}^{a}\frac{1}{\sqrt{2\pi}} \exp\left( -\frac{z^{2}}{2} \right)dz\label{eq:transferFunction} \textrm{.}
    \end{equation}
    \item $a_{j}^{l}$ -- Input to the transfer function for neuron~$j$ in layer~$l$.  It is defined as:
    \begin{equation}
      a_{j}^{l} := \w_{j}^{l} \xvec_{j}^{l} + b_{j}^{l}\text{rm}.
    \end{equation}
    \item $J$ -- Squared loss function.  It is defined formally as 
    \begin{equation}
      J = \frac{1}{2} (\yhat -y)^2
    \end{equation}
  \end{itemize}
  
  \textbf{Additional Notes:} By the Fundamental Theorem of Calculus, for given a continuous function~$f$, the derivative of its definite integral whose lower limit is a constant is given by
  
  \begin{equation}
    \frac{\partial}{\partial x}\int_{a}^{x}f(t)dt=f(x)
  \end{equation}
  
  The transfer function in Eq.~\ref{eq:transferFunction} is not exactly in this form.  However, it could be transformed into that form via a limit as shown below.
  
  \begin{equation}
    \frac{\partial}{\partial a} \Phi(a) = \frac{\partial}{\partial a} \int_{-\infty}^{a}f(t)dt =\underset{b\rightarrow -\infty}\lim \frac{\partial}{\partial x} \int_{b}^{x}f(t)dt=\frac{1}{\sqrt{2\pi}} \exp\left( -\frac{x^{2}}{2} \right)
  \end{equation}
  
  \textbf{Back-propagation for the Weights}: Using the chain rule, the back-propagation for the output layer is:
  
  \begin{equation}
    \frac{\partial J}{\partial \w_{j}^{2}} = \frac{\partial J}{\partial \yhat} \cdot \frac{\partial \yhat}{\partial \w_{1}^{3}}  
  \end{equation}
  
  The loss function for this network is given by the squared loss.  Given a single target value~$y$, the derivative of this function is:
  
  \begin{align}
    \frac{\partial J}{\partial \yhat} &= \frac{\partial}{\partial \yhat} \left(\frac{1}{2} \left( \yhat - y\right)^2 \right) = \yhat - y \label{eq:derivLoss}
  \end{align}
  
  The derivative of~$\yhat$ is then
  
  \begin{align}
    \frac{\partial}{\partial \w_{1}^{3}} \yhat = \frac{\partial}{\partial \w_{1}^{3}}\Phi(a_{1}^{3})=\frac{1}{\sqrt{2\pi}} \exp\left( -\frac{\left(\left(\w_{1}^{3}\right)^\T\xvec_{1}^{3}+b_{1}^{3}\right)^{2}}{2} \right)\xvec_{1}^{3}\textrm{.}
  \end{align}
  
  This makes the complete derivative for the output layer is:
  
  \begin{equation}
    \frac{\partial L}{\partial \w_{1}^{3}} = \boxed{\left( \yhat - y \right) \cdot \frac{1}{\sqrt{2\pi}} \exp\left( -\frac{\left(\left(\w_{1}^{3}\right)^\T\xvec_{1}^{3}+b_{1}^{3}\right)^{2}}{2} \right)\xvec_{1}^{3}} \label{eq:outputLayerDerivative}
  \end{equation}
    
  The derivative of the $j^{th}$~neuron in the hidden layer is then found via:
  
  \begin{align}
    \frac{\partial J}{\partial \w_{j}^{2}} &= \frac{\partial J}{\partial \yhat} \cdot \frac{\partial \yhat}{\partial x_{1}^{3}} \cdot \frac{\partial x_{1}^{3}}{\partial \w_{j}^{2}}\label{eq:hiddenLayerChainRule}\\
      &=\left(\yhat - y\right) \cdot \frac{1}{\sqrt{2\pi}} \exp\left( -\frac{\left(\left(\w_{1}^{3}\right)^\T\xvec_{1}^{3}+b_{1}^{3}\right)^{2}}{2} \right)\w_{1}^{3} \cdot \frac{\partial x_{1}^{3}}{\partial \w_{j}^{2}}\\
      &=\boxed{\left(\yhat - y\right) \cdot \frac{1}{\sqrt{2\pi}} \exp\left( -\frac{\left(\left(\w_{1}^{3}\right)^\T\xvec_{1}^{3}+b_{1}^{3}\right)^{2}}{2} \right)\w_{1}^{3} \cdot \frac{1}{\sqrt{2\pi}} \exp\left( -\frac{\left(\left(\w_{j}^{2}\right)^\T\xvec_{j}^{2}+b_{j}^{2}\right)^{2}}{2} \right)\xvec_{j}^{2}}
      \textrm{.}
  \end{align}

  
  \textbf{Back-propagation for the Bias Terms}: Taking the derivative with respect to the bias is simpler.  Eq.~\ref{eq:derivLoss} applies as it did for the weights.  Now, the derivative of the transfer function with respect to~$b_{1}^{3}$ is:
  
  \begin{align}
    \frac{\partial}{\partial b_{1}^{3}} \yhat = \frac{\partial}{\partial b_{1}^{3}}\Phi(a_{1}^{3})=\frac{1}{\sqrt{2\pi}} \exp\left( -\frac{\left(\left(\w_{1}^{3}\right)^\T\xvec_{1}^{3}+b_{1}^{3}\right)^{2}}{2} \right)\textrm{.}
  \end{align}
    
  This makes the bias term derivative for the output layer is 
    
  \begin{align}
    \frac{\partial}{\partial b_{1}^{3}} L =\boxed{\left(\yhat - y\right)\frac{1}{\sqrt{2\pi}} \exp\left( -\frac{\left(\left(\w_{1}^{3}\right)^\T\xvec_{1}^{3}+b_{1}^{3}\right)^{2}}{2} \right)}\textrm{.}
  \end{align}
  
  Similar to what was done in the last equation and Eq.~\ref{eq:hiddenLayerChainRule}, we can find the bias term for the $j^{\text{th}}$~neuron in the hidden layer.  Hence, it is:
  
    \begin{align}
  \frac{\partial J}{\partial b_{j}^{2}} &= \frac{\partial J}{\partial \yhat} \cdot \frac{\partial \yhat}{\partial \xvec_{1}^{3}} \cdot \frac{\partial \xvec_{1}^{3}}{\partial b_{j}^{2}}\\
    &=\left(\yhat - y\right) \cdot \frac{1}{\sqrt{2\pi}} \exp\left( -\frac{\left(\left(\w_{1}^{3}\right)^\T\xvec_{1}^{3}+b_{1}^{3}\right)^{2}}{2} \right)\w_{1}^{3} \cdot \frac{\partial x_{1}^{3}}{\partial b_{j}^{2}}\\
    &=\boxed{\left(\yhat - y\right) \cdot \frac{1}{\sqrt{2\pi}} \exp\left( -\frac{\left(\left(\w_{1}^{3}\right)^\T\xvec_{1}^{3}+b_{1}^{3}\right)^{2}}{2} \right)\w_{1}^{3} \cdot \frac{1}{\sqrt{2\pi}} \exp\left( -\frac{\left(\left(\w_{j}^{2}\right)^\T\xvec_{j}^{2}+b_{j}^{2}\right)^{2}}{2} \right)}\textrm{.}
  \end{align}
  
  
  
  %---------------------------------------------------%
  \newpage
  \begin{problem}
    Derive the matching loss for the \textit{rectifier} activation/transfer function~${f(a) := \max(0,a)}$.  This function is also known as the ramp function.
    \problemspace
    Hint: Review how the matching loss is computed when the transfer functions are the sigmoid functions and the sign function:~${f(a) = \sign(a)}$. (See material for lecture 5.)
  \end{problem}
  
  The $\max$~function can be written as a piecewise identity as shown below.
  
  \begin{equation}
    \max(0,x)=\left\{
                \begin{array}{ll}
                  x & x \geq 0\\
                  0 & \text{otherwise}
                \end{array}
              \right.
  \end{equation}
 
  \noindent
  The integral of the $\max$~function is similarly piecewise.

  \begin{align}
     \int \max(0,x) dx = H(x) &=  \left\{
                                    \begin{array}{cl}
                                      \frac{x^2}{2} & x \geq 0\\
                                      0 & \text{otherwise}
                                    \end{array}
                                  \right.\\
                              &=  \frac{1}{2}\max(0,x)x
  \end{align}

  For a function~$H$, the matching loss is defined as
  
  \begin{align}
    \Delta_{H}(\wx, y) = H(\wx) - H(y) - (\wx - y)h(y)\textrm{.}
  \end{align}

  Substituting, the matching loss becomes
  
  \begin{align}
    \boxed{\Delta_{H}(\wx, y) = \frac{1}{2}\left( \max(0,\wx)\wx - \max(0,y)y - 2(\wx -y)\max(0,y) \right)}
  \end{align}

\end{document}