\documentclass{report}

\usepackage{fullpage}
\usepackage[skip=4pt]{caption} % ``skip'' sets the spacing between the figure and the caption.
\usepackage{pgfplots}   % Needed for plotting
\usepackage{amsmath}    % Allows for piecewise functions using the ``cases'' construct
%\usepackage{mathrsfs}   % Use the ``\mathscr'' command in an equation.

\usepackage[obeyspaces,spaces]{url} % Used for typesetting with the ``path'' command
\usepackage[hidelinks]{hyperref}   % Make the cross references clickable hyperlinks
\usepackage[bottom]{footmisc} % Prevents the table going below the footnote
\usepackage{nccmath}    % Needed in the workaround for the ``aligncustom'' environment
\usepackage{amssymb}    % Used for black QED symbol   
\usepackage{bm}    % Allows for bolding math symbols.

\usepackage{tabto}     % Allows to tab to certain point on a line

\newcommand{\hangindentdistance}{1cm}
\setlength{\parindent}{0pt}
\setlength{\leftskip}{\hangindentdistance}
\setlength{\hangafter}{1}


% Set up page formatting
\usepackage{fancyhdr} % Used for every page footer and title.
\pagestyle{fancy}
\fancyhf{} % Clears both the header and footer
\renewcommand{\headrulewidth}{0pt} % Eliminates line at the top of the page.
\fancyfoot[LO]{CMPS242 \textendash{} Homework \#4} % Left
\fancyfoot[CO]{\thepage} % Center
\fancyfoot[RO]{Zayd Hammoudeh} %Right

% Change interline spacing.
\renewcommand{\baselinestretch}{1.1}
\newenvironment{aligncustom}
{ \csname align*\endcsname % Need to do this instead of \begin{align*} because of LaTeX bug.
    \centering
}
{
  \csname endalign*\endcsname
}
%--------------------------------------------------


\title{\textbf{CMPS242 Homework \#4}}
\author{Zayd Hammoudeh}

%---------------------------------------------------%
% Define the Environments for the Problem Inclusion %
%---------------------------------------------------%
\newcounter{problemCount} 
\setcounter{problemCount}{0} % Reset the subproblem counter
\newenvironment{problemshell}{
  \par%
  \medskip
  \leftskip=0pt\rightskip=0pt%
}
{
  \par\medskip
}
\newenvironment{problem}
{%
  \stepcounter{problemCount}
  \begin{problemshell}
    \noindent \textit{Problem \#\arabic{problemCount}} \\
    \bfseries  
}
{
  \end{problemshell}
}


\newcommand{\problemspace}{\\[0.4em]}
\newcommand{\sign}{\text{\normalfont sign}}


\begin{document}
  \maketitle
  
  \begin{problem}
    Consider $1$-dimensional linear regression.
    \problemspace
    First compute the optimum solution,~$w^{*}$, for a batch of examples,~${(x_i,y_i)}$, ${1 \leq n \leq n}$,~ie the weight that minimizes the total loss on examples:~${L(w)=\sum_{i=1}^n(wx_{i}-y_{i})^2}$.
    \problemspace
    Assume labels are expensive (see lecture~7).  You are given only one of the labels~$y-i$.  Compute the optimal solution~$w_i$ based on a single example~$(x_i,y_i)$.
    \problemspace
    Show that if~$i$ is chosen wrt the distribution~$\frac{x_{i}^{2}}{\sum_{j}x_{j}^{2}}$, then the expected loss of~$w_{i}$ on all examples is twice the optimum ie
    \[\mathbf{E}[L(w_i)] = 2L(w^{*}), \]
    \noindent
    when all~$x_i$ are non-zero.
    \problemspace
    Hint: First check the above equation on Octave or Matlab on some random data.  Make your solution as simple as you can.
  \end{problem}
  




  %---------------------------------------------------%
  \newpage
  \begin{problem}
    Compute all the derivatives using Backpropagation for a 3-layer neural net with one output when the transfer function is the cumulative Gaussian density
    \[\Phi(a) = \int_{-\infty}^{a}\frac{1}{\sqrt{2\pi}} \exp\left( -\frac{z^2}{2}\right) dz\]
    \noindent
    and the output node is the square loss.  Assume the node of the hidden layer as well as the output node each have a bias term.  Compute the derivatives of the loss wrt the weights between the 2nd~and 3rd~layer and the 1st~and 2nd~layer as well as the derivatives of the loss wrt the bias terms.
    \problemspace
    Hint: First produce a writeup when the transfer function is the sigmoid and then modify it.
  \end{problem}






  %---------------------------------------------------%
  \newpage
  \begin{problem}
    Derive the matching loss for the \textit{rectifier} activation/transfer function~${f(a) := \max(0,a)}$.  This function is also known as the ramp function.
    \problemspace
    Hint: Review how the matching loss is computed when the transfer functions are the sigmoid functions and the sign function:~${f(a) = \sign(a)}$. (See material for lecture 5.)
  \end{problem}
  
 
 

\end{document}